\section{Mathematical Induction}

To prove the statement $P(n)$ is true for all $n = 1,2,3,...$, we must show:

    \begin{enumerate}
        \item Base Case Step: Verify that $P(1)$ is true
        \item Inductive Step: Show the conditional statement $P(k) \implies P(k+1)$ is true for all $k = 1,2,3,...$
        \item Conclusion: $P(n)$ is true for all $n=1,2,3,...$
    \end{enumerate}

\begin{problem}
    Prove that $1+2+3+...+n=\frac{n(n+1)}{2}$ is true for all positive integers $n$ using mathematicl induction.

    \begin{proof}
        \textbf{BASE CASE:} If $n=1$, we see that $1 = \frac{1(1+1)}{2}=\frac{2}{2}$.\\

        \textbf{INDUCTION STEP:} Suppose for some $k \in \Z^+$, that $1+2+3+...+k=\frac{k(k+1)}{2}$. Then, 

        $$
        \begin{aligned}
            1+2+3+...+k+(k+1)&=\frac{k(k+1)}{2}+(k+1)\\
            &= \frac{k(k+1)}{2} + \frac{2(k+1)}{2}\\
            &=\frac{k(k+1)+2(k+1)}{2}\\
            &=\frac{(k+1)(k+2)}{2} = \frac{(k+1)((k+1)+1)}{2}
        \end{aligned}
        $$

        \textbf{CONCLUSION:} By induction, we know that for any $n \in \Z^+$, $1+2+3+...+n=\frac{n(n+1)}{2}$.
    \end{proof}
\end{problem}\newpage


\begin{problem}
    Use mathematical induction to prove that for $n\in\Z$, $2^n < n!$ for all $n > 3$.

    \begin{proof}
        \textbf{BASE CASE:} If $n = 4$, then $2^4=16<4!=24$.\\

        \textbf{INDUCTION STEP;} Suppose that for some $k\in\Z$  and $k > 3$ that $2^k < k!$. Then:

        $$
        \begin{aligned}
            2^{k+1} < (k+1)!\\
            2(2^k) < (k+1)k!
        \end{aligned}
        $$

        Since $k\ge 4$, the factor $(k+1)>2$, the factor being added with the next term will be greater than the factor of $2$ being added on the left side, which means that the righ side will always be greater than the left side.\\

        \textbf{CONSLLUSION:} By induction, for $n\in\Z$, $2^n < n!$ for all $n > 3$.
    \end{proof}
\end{problem}


\section{Strong Induction}

To prove the statement $P(n)$ is true for all $n=1,2,3,...$, we show the following:
    
    \begin{enumerate}
        \item Base Case Step: Verify that $P(1)$ is true
        \item Inductive Step: Show that the conditional statement $$[P(1) \wedge P(2) \wedge... \wedge P(k)] \implies P(k+1)$$ is true for all $k = 1,2,3,...$
        \item Conclusion: $P(n)$ is true for all $n=1,2,3...$
    \end{enumerate}

\begin{problem}
    Prove that every integer greater than one can be written as the product of primes.

    \begin{proof}
        \textbf{BASE CASE:} If $n = 2$, then since $2$ is prime, it is trivially a product of two primes (being 2 and 1).\\

        \textbf{INDUCTION STEP:} Suppose for some $k \in \Z$ that all integers between 2 and $k-1$ can be written as a product of primes. If $k$ is prime, then it is trivially a product of two primse (being $k$ and 1). If $k$ is not prime, then $k = ab$ where $a$ and $b$ are between $2$ and $k-1$. Then $a$ and $b$ can be written as a product of primes. Since $k=ab$, then $k$ can be written as a product of primes. \\

        \textbf{CONCLUSION:} Therefore by strong induction, we know that every integer greater than one can be written as the product of two primes.
    \end{proof}
\end{problem}