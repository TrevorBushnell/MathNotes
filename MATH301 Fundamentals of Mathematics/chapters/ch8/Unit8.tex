When we are proving statements, we have no clue whether a given statement is true or not. As such, when we try to write complex proofs, the two mains strategies we can use are \textbf{start ith an interesting example/easy proposition and try to generalize} OR \textbf{make a guess at a theorem (conjecture) and try to either prove or disproof that theorem}.\\

A \textbf{disproof} is a proof that proves that the statement is false. In other words, a disproof is a proof of the negation of the original statement.


\section{Counterexamples}

Since many theorems have a "for all" statement or an "implies" statement, then the negation becomes an existence statement. As such, we can write a constructive proof that shows that the negation of the statement is true. This is called a \textbf{counterexample}. In other words, \textit{you are trying to come up with one example that shows the original statement to be false.}

