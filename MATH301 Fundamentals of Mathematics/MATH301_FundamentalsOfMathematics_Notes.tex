
\documentclass{package/notes}
\usepackage[english]{babel}
\usepackage{amssymb,amsmath,amsfonts}  %%% for maths
%%%%%%%%%%%%%%%%%%%%%%%%%%%%%%%%%%%%%
\usepackage{package/color-env}
\usepackage{lipsum}

\newcommand{\Z}{\mathbb{Z}}
\newcommand{\N}{\mathbb{N}}
\newcommand{\R}{\mathbb{R}}
\newcommand{\Q}{\mathbb{Q}}
\newcommand{\C}{\mathbb{C}}
\newcommand{\lcm}{\text{lcm}}
\renewcommand\qedsymbol{$\blacksquare$}
%%%%%%%%%%%%%%%%%%%%%%%%%%%%%%%%%%%%%

\begin{document}

	\begin{titlepage} % Suppresses headers and footers on the title page
		
		\centering % Centre everything on the title page
		
		\scshape % Use small caps for all text on the title page
		
		\vspace*{\baselineskip} % White space at the top of the page
		
		%------------------------------------------------
		%	Title
		%------------------------------------------------
		
		\rule{\textwidth}{1.6pt}\vspace*{-\baselineskip}\vspace*{2pt} % Thick horizontal rule
		\rule{\textwidth}{0.4pt} % Thin horizontal rule
		
		\vspace{0.75\baselineskip} % Whitespace above the title
		
		{\huge MATH301: Fundamentals of Mathematics Notes\\} % Title
		
		\vspace{0.75\baselineskip} % Whitespace below the title
		
		\rule{\textwidth}{0.4pt}\vspace*{-\baselineskip}\vspace{3.2pt} % Thin horizontal rule
		\rule{\textwidth}{1.6pt} % Thick horizontal rule
		
		\vspace{2\baselineskip} % Whitespace after the title block
		
		%------------------------------------------------
		%	Subtitle
		%------------------------------------------------
		 
		
		\vspace*{3\baselineskip} % Whitespace under the subtitle
		
		
		
		\vspace{0.5\baselineskip} 
		
		\vspace{0.5\baselineskip} 
		
		\vfill 
		
		%------------------------------------------------
		% Author
		%------------------------------------------------
		
		
		\vspace{0.3\baselineskip} 
		
		
		{\large Edited by\\  Trevor Bushnell} 
		
	\end{titlepage}
	\tableofcontents
%\newpage

\chapter*{Introduction}

This document aims to highlight the important content of the MATH301 course in traditional notes format. These notes are completely open-source, which means anyone is allowed to use these notes for their own personal benefit without having to seek permission from myself. \newline

Due to the open-source nature of these notes, anyone is allowed to contribute to improving these notes as they see fit. Since I am using \LaTeX to write these notes and I am using GitHub to distribute these notes easily, you must request all changes through the repository website on GitHub, which you can find \textbf{here}. If you are interested in contributing to these notes, then there are a few ways that you can do so:\newline

\begin{enumerate}
	\item \textbf{Open and submit an issue on my GitHub repository:} I write all my notes in \LaTeX, which is a typesetting language that is really helpful when it comes to typing and rendering math equations quickly and easily. If you do not know how to write \LaTeX code but are still interested in making a change to the notes, you can open an issue by going to the MathNotes repo on GitHub, and clicking on the button labeled "New Issue." From there, you can type out the change that you wish to see in the notes. It would be helpful if you would indicate what course you would like to see changed so that I can understand what you are referring to. I will then update the code to include your issue so that you don't have to worry about writing the code yourself.
	\item \textbf{Create and submit a pull request:} If you know how to write LaTeX code and you understand how GitHub works, you can submit a pull request where you can write the code that you want to change yourself. I will then review the code and either submit the code to be incorporated into the notes OR provide some comments on your code if I wish for something to be different. 
\end{enumerate}

Thank you so much for using these notes. I hope that the information is provided in such a way that it can help you when reviewing content for you AP test/class exam and just in general when it comes to learning the content for the course. Happy studying!

\chapter{Set Theory}
\section{Sets}

\begin{definition}[Sets and Elements]{def1.1:label}
    A \textbf{set} is a collection of \textit{elements}.\\

    $A = C$ means that the sets $A$ and $C$ both have the same elements (doesn't mean that the elements are in the same order).\\

    Examples of sets include the following:

    \begin{equation*}
        \begin{aligned}
            &\{1,2,3,4,5\}\\
            &\{A,B,C,D,E,F\}
        \end{aligned}
    \end{equation*}

    Elements don't have to be numerical, they can be letters, words, phrases, numbers, etc.
\end{definition}

\begin{definition}[Cardinality]{def1.1.2:label}
    The \textbf{cardinality} of a finite set is the number of elements in the set, denoted $|A|$. \\

    For example, for the set $A = \{1,2,3\}$, the cardinality of the set is $3$ because there are three elements in the set.
\end{definition}


\begin{itemize}
    \item The \textbf{empty set} ($\varnothing$) is the set that contains no elements (so $\varnothing = \{\}$)
    \item A lot of time when we are writing in sets, we prefer to write in \textbf{set builder notation}, which means that inside the brackets we write the types of elements that are in the set, and then define a rule which the elements in the set follow
    \begin{itemize}
        \item EXAMPLE: $\{x : x \ge 2\}$ means the set containing any numerical element such that the element is greater than or equal to 2.
    \end{itemize}
\end{itemize}\newpage


\section{The Cartesian Product}

\begin{definition}[Ordered Pairs]{def1.2.1:label}
    An \textbf{ordered pair} is a list $(x,y)$ of two elements $x$ and $y$, enclosed in parenthases and separated by a comma. 
\end{definition}


\begin{definition}[Cartesian Product]{def1.2.2:label}
    The \textbf{Cartesian product} of two sets $A$ ad $B$ is the set $A \times B$ or all ordered pairs with the first element from $A$ and the second element from $B$. \\

    $$
    A \times B = \{(a,b) : a \in A \:\&\: b \in B\}
    $$
\end{definition}


\begin{problem}
    Find the cross product of the sets $A = \{1,2,3\}$ and $B = \{c,d\}$.

    $$
    A \times B = \{(1,c), (1,d),(2,c), (2,d),(3,c), (3,d)\}
    $$
\end{problem}



\chapter{Logic}
\input{chapters/ch2/Unit2.tex}



\chapter{Direct Proofs}
\section{Integers and Divisibility}

While $\Z$ is not closed under division (an integer divided by an integer is not always an integer), we can still come up with definitions for dividing integers. 

\begin{definition}{def3.1}
    If $a,b\in\Z$ and $a\ne 0$, then $a$ \textbf{divides} $b$ ($a|b$) if there exists an integer $k$ such that $ak = b$. In this case, $a$ is a \textbf{factor/divisor} of $b$, and $b$ is a \textbf{multiple} of $a$. If $a$ does not divide $b$, then the notation is $a \nmid b$.
\end{definition}

\begin{itemize}
    \item $2|10$ because $2(5)=10$ where $k=5$
    \item $4 \nmid 10$ because $\frac{10}{4} = 2.5$ which is not an integer
    \item $-6|42$ because $-6(-7)=42$ where $k=-7$
\end{itemize}


\begin{theorem}[The Divisibility Theorem]{theorem3.1:label}
    Let $a,b,c \in \Z$ where $a \ne 0$. Then the following statements are true:

    \begin{itemize}
        \item $a|a$
        \item $a|0$
        \item If $a|b$ and $a|c$, then $a|(b+c)$
        \item If $a|b$, then $a|bc$ for all integers $c$
        \item If $a|b$ and $b|c$, then $a|c$
    \end{itemize}
\end{theorem}

\begin{corollary}[Corollary to the Divisibility Theorem]{cor3.1:label}
    Let $a,b,c\in \Z$ where $a\ne 0$. If $a|b$ and $b|c$, then $a | (mb + nc)$ for all integers $m$ and $n$.
\end{corollary}

\subsection{The Division Algorithm}

\begin{itemize}
    \item Let $a \in \Z$ and $d \in \Z^+$
    \item There are unique integers $q$ and $r$ such that $a = dq + r$ where $0 \le r < d$
    \item $a$ is the \textit{dividend}
    \item $d$ is the \textit{divisor}
    \item $q$ is the \textit{quotient}
    \item $r$ is the remainder 
\end{itemize}

\textbf{THE $\div$ OPERATION:} $a \div b$ means divide $b$ by $a$ and leave off the remainder. 

\begin{itemize}
    \item EX: $4 \div 10 = 2$ because $\frac{10}{4} = 2.5$ and we just care about the whole number part of the answer and disregard the remainder.
\end{itemize}

\textbf{THE $\mod$ OPERATOR:} $a \mod b$ means divide $b$ by $a$ and leave JUST the remainder.

\begin{itemize}
    \item EX: $4 \mod 10 = 2$ because $\frac{10}{4} = 2\text{ R}2$ and we just care about the remainder part of the answer (2)
\end{itemize}

\begin{problem}
    If 100 is divided by 23, write as $a=dq+r$ and find the quotient and remainder.

    $$
    \begin{aligned}
        100 &= 23q + r\\
        100 &= 23 \cdot 4 + 8\\
        100 \div 23 = 4\\
        100 \mod 23 = 8
    \end{aligned}
    $$
\end{problem}

\begin{problem}
    If -100 is divided by 23, write as $a=dq+r$ and find the quotient and remainder.

    $$
    \begin{aligned}
        -100 &= 23q + r
        -100 &= 23(-4) + (-8)\\
        \text{remainder can't be negative so we add another factor of 23 to make the remanider positive}
        -100 &= 23(-5) + 15\\
        -100 \div 23 = -5\\
        -100 \mod 23 = 15
    \end{aligned}
    $$
\end{problem}


\subsection{Introduction to Congruence}

\begin{definition}[Modular Congruence]{def3.2:label}
    Let $a$ and $b$ be integers and let $n$ be a positive integer. Then $a$ and $b$ are \textbf{congruent modulo $n$} if $n|(a-b)$. If $a$ and $b$ are congruent moduo $n$, we then write that $a \equiv b (\mod n)$
\end{definition}

\textbf{NOTE:} $a$ and $b$ are congruent modulo $n$ when the have the same remainder when divided by $n$. This means that if $a$ and $b$ are congruent modulo $n$, then $a \mod n = b \mod n$.

\begin{itemize}
    \item $12 \equiv 2 \mod 5$ because $12 \mod 5 = 2 = 2 \mod 5$
    \item $-7 \not\equiv 2 \mod 5$ because $-7 \mod 5 = 3 \ne 2 \mod 5$
\end{itemize}

\textbf{NOTE:} $\{5k + 2 | k \in \Z\}$ is the set of numbers congruent to a modulo $n$



\section{Integer Representations and Algorithms}

\subsection{Integers in Different Bases}

\begin{theorem}[Integers in Differen Bases]{theorem3.2:label}
    Let $b \in \Z$ and $b > 1$. Then if $n \in \Z^+$, it has a unique base-$b$ expansion, expressed uniquely in the form:

    $$
    n = a_kb^k + a_{k-1}b^{k+1} + \cdots + a_1b^1 + a_0
    $$

    Where $k$ is a non-negative integer and $a_0,a_1,a_2,\cdots,a_k$ are all non-negative integers that are less than $b$ and $a_k \ne 0$.
\end{theorem}

\begin{itemize}
    \item We denote what base we are in by using a subscript equivalent to the base: $(a_ka_{k-1}...a_1a_0)_b$
    \item $b = 10$ is \textit{decimal} expansion
    \item $b = 2$ is \textit{binary} expansion
    \item $b=8$ is the \textit{octal} expansion
    \item $b = 16$ is the \textit{hexaedcimal} expansion
\end{itemize}

\begin{problem}
    Convert $(10010111)_2$ to decimal.

    $$
    \begin{aligned}
        (10010111)_2 &= 1 \cdot 2^7 + 0 \cdot 2 ^6 + 0 \cdot 2 ^5 + 1 \cdot 2 ^4 + 0 \cdot 2 ^3 + 1 \cdot 2 ^2 + 1 \cdot 2 ^1 + 1 \cdot 2 ^0\\
        &= 128 + 16 + 4 + 2 + 1\\
        &= 151
    \end{aligned}
    $$
\end{problem}


\begin{problem}
    Convert $(A1F)_{16}$ to decimal.
\end{problem}


\begin{problem}
    Convert $(1036)_8$ to decimal.
\end{problem}


\subsection{Converting From Base-10 to Base-$n$}


To convert a number from base-10 to base-$n$, you repeatedly use the division algorithm in the following way:

$$
\begin{aligned}
    n &= bq_0 + r_0\\
    q_0 &= bq_1 + r_1\\
    q_1 &= bq_2 + r_2\\
    &...\\
    q_{k-1} = bq_k + r_k
\end{aligned}
$$

You stop this process when $q_k$ is equal to 0. Then, the number in base $n$ is the list of all $r_n$ (all the remainders) in reverse order (so from the bottom step remainder to the first step remainder).

\begin{problem}
    Convert $(108)_10$ into the following bases:

    \begin{itemize}
        \item Binary:
        \item Octal:
        \item Hexadecimal:
    \end{itemize}
\end{problem}


\subsection{Converting From Base-$n$ to Base-$m$}

Every three digits in binary correspond to one octal digit. Similarly, every four digits in binary correspond to one hexadecimal digit.\\

On the next page, you can see a chart matching all of the base 10 numbers from 0-16 and what each number's binary, octal, and hexadecimal representations are. You can then use this table to convert any number in base 2,4,8,16 to base 2,4,8,16.\newpage

\begin{longtable}[]{@{}llll@{}}
    \toprule
    Decimal & Binary & Octal & Hexadecimal \\ \addlinespace
    \midrule
    \endhead
    0 & 0 & 0 & 0 \\ \addlinespace
    1 & 1 & 1 & 1 \\ \addlinespace
    2 & 10 & 2 & 2 \\ \addlinespace
    3 & 11 & 3 & 3 \\ \addlinespace
    4 & 100 & 4 & 4 \\ \addlinespace
    5 & 101 & 5 & 5 \\ \addlinespace
    6 & 110 & 6 & 6 \\ \addlinespace
    7 & 111 & 7 & 7 \\ \addlinespace
    8 & 1000 & 10 & 8 \\ \addlinespace
    9 & 1001 & 11 & 9 \\ \addlinespace
    10 & 1010 & 12 & A \\ \addlinespace
    11 & 1011 & 13 & B \\ \addlinespace
    12 & 1100 & 14 & C \\ \addlinespace
    13 & 1101 & 15 & D \\ \addlinespace
    14 & 1110 & 16 & E \\ \addlinespace
    15 & 1111 & 17 & F \\ \addlinespace
    16 & 10000 & 20 & 10 \\ \addlinespace
    \bottomrule
\end{longtable}

\begin{problem}
    Convert $(DB5)_{16}$ into binary and octal.
\end{problem}


\section{Primes and Greatest Common Divsors}

\begin{definition}[Prime Numbers]{def3.3.1:label}
    An integer $p$ greater than 1 is \textbf{prime} if the only positive factors of $p$ are 1 and $p$.
\end{definition}

\begin{theorem}[The Fundamental Theorem of Arithmetic]{theorem3.3.1:label}
    Every integer greater than 1 can be written uniquely as a prime or as the prodct of two or more primes written in increasing order.
\end{theorem}

\begin{problem}
    Fnid the prime factorizations of the following numbers:

    \begin{itemize}
        \item $36 = 6^2 = (2\cdot 3)^2 = 2^2\cdot 3^2$\\
        \item $84 = 7 \cdot 12 = 7 \cdot 3 \cdot 2 \cdot 2 = 2^2\cdot 3 \cdot 7$
        \item $31 = 31 \cdot 1$
    \end{itemize}
\end{problem}

\begin{proposition}{prop3.3.1:label}
    If $n$ has a prime divisor, then it must have one that is less than or equal to $\sqrt{n}$
\end{proposition}

Listed below is a table with all of the primes that are $\le 100$. Every number that is \textbf{bold} is a prime number.

\begin{longtable}[]{@{}llllllllll@{}}
    \toprule
    1 & \textbf{2} & \textbf{3} & 4 & \textbf{5} & 6 & \textbf{7} & 8 & 9 &
    10 \\ \addlinespace
    \textbf{11} & 12 & \textbf{13} & 14 & 15 & 16 & \textbf{17} & 18 &
    \textbf{19} & 20 \\ \addlinespace
    21 & 22 & \textbf{23} & 24 & 25 & 26 & 27 & 28 & \textbf{29} &
    30 \\ \addlinespace
    \textbf{31} & 32 & 33 & 34 & 35 & 36 & \textbf{37} & 38 & 39 &
    40 \\ \addlinespace
    \textbf{41} & 42 & \textbf{43} & 44 & 45 & 46 & \textbf{47} & 48 & 49 &
    50 \\ \addlinespace
    51 & 52 & \textbf{53} & 54 & 55 & 56 & 57 & 58 & \textbf{59} &
    60 \\ \addlinespace
    \textbf{61} & 62 & 63 & 64 & 65 & 66 & \textbf{67} & 68 & 69 &
    70 \\ \addlinespace
    \textbf{71} & 72 & \textbf{73} & 74 & 75 & 76 & 77 & 78 & \textbf{79} &
    80 \\ \addlinespace
    81 & 82 & \textbf{83} & 84 & 85 & 86 & 87 & 88 & \textbf{89} &
    90 \\ \addlinespace
    91 & 92 & 93 & 94 & 95 & 96 & \textbf{97} & 98 & 99 &
    100 \\ \addlinespace
    \bottomrule
    \end{longtable}


\begin{theorem}{3.3.2:label}
    There are infinietly many primes.
\end{theorem}

\begin{definition}{3.3.2:label}
    Let $a$ and $b$ be integers with $a$ and $b$ not being equal to 0. The largest integer $d$ such that $d|a$ and $d|b$ is called the \textbf{greatest common divisor} of $a$ and $b$. The notation is $\gcd(a,b)$
\end{definition}

\begin{problem}
    Find $\gcd(36,84).$

    $$
    \begin{aligned}
        36 &= 6 \cdot 6 = 2^2 \cdot 3^2\\
        84 &= 7 \cdot 12 = 2^2 cdot 3 \cdot 7\\
        \gcd(36,84) &= 2^2 \cdot 3 = 12
    \end{aligned}
    $$
\end{problem}

\begin{problem}
    Find $\gcd(9,49).$

    $$
    \begin{aligned}
        9 &= 3^2\\
        49 &= 7^2\\
        \gcd(9,49) &= 1
    \end{aligned}
    $$
\end{problem}


\begin{definition}[Relatively Prime]{def3.3.3:label}
    Two integers $a$ and $b$ are \textbf{relatively prime} if $\gcd(a,b) = 1$.
\end{definition}


\begin{definition}{def3.3.4:label}
    The least common multiple of the positive integers $a$ and $b$ is is the smallest positive integer that is divisible by both $a$ and $b$. Notation is $\lcm(a,b)$.
\end{definition}








\end{document}
.