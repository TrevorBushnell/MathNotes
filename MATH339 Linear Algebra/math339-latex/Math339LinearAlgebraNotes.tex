
\documentclass{package/notes}
\usepackage[english]{babel}
\usepackage{amssymb,amsmath,amsfonts}  %%% for maths
%%%%%%%%%%%%%%%%%%%%%%%%%%%%%%%%%%%%%
\usepackage{package/color-env}
\usepackage{lipsum}
\renewcommand\qedsymbol{$\blacksquare$}
%%%%%%%%%%%%%%%%%%%%%%%%%%%%%%%%%%%%%

\begin{document}

	\begin{titlepage} % Suppresses headers and footers on the title page
		
		\centering % Centre everything on the title page
		
		\scshape % Use small caps for all text on the title page
		
		\vspace*{\baselineskip} % White space at the top of the page
		
		%------------------------------------------------
		%	Title
		%------------------------------------------------
		
		\rule{\textwidth}{1.6pt}\vspace*{-\baselineskip}\vspace*{2pt} % Thick horizontal rule
		\rule{\textwidth}{0.4pt} % Thin horizontal rule
		
		\vspace{0.75\baselineskip} % Whitespace above the title
		
		{\huge Math 339: Linear Algebra\\} % Title
		
		\vspace{0.75\baselineskip} % Whitespace below the title
		
		\rule{\textwidth}{0.4pt}\vspace*{-\baselineskip}\vspace{3.2pt} % Thin horizontal rule
		\rule{\textwidth}{1.6pt} % Thick horizontal rule
		
		\vspace{2\baselineskip} % Whitespace after the title block
		
		%------------------------------------------------
		%	Subtitle
		%------------------------------------------------
		
		\LARGE{Public Notes for Any Linear Algebra Course} 
		
		\vspace*{3\baselineskip} % Whitespace under the subtitle
		
		
		
		\vspace{0.5\baselineskip} 
		
		
		
		\vspace{0.5\baselineskip} 
		
		
		
		\vfill 
		
		%------------------------------------------------
		% Author
		%------------------------------------------------
		
		
		\vspace{0.3\baselineskip} 
		
		
		{\large Edited by\\  Trevor Bushnell} 
		
	\end{titlepage}
	\tableofcontents
%\newpage
\chapter{Linear Equations}

\section{Systems of Linear Equations}

\section{Row Reduction and Echelon Form}

\section{Vector Equations}

\section{The Matrix Equation}

\section{Solution Sets of Linear Systems}

\section{Linear Independence}

\section{Introduction to Linear Transformations}

\section{The Matrix of a Linear Transformation}



\chapter{Matrix Algebra}

\section{Matrix Operations}

\section{The Inverse of a Matrix}

\section{Characterizations of Invertible Matrices}

\section{Partitioned Matrices}

\section{Matrix Factorizations}

\section{The Leontief Input-Output Model}

\section{Applications to Computer Graphics}

\section{Subspaces of $R^n$}




\chapter{Determinants}

\section{Introduction to Determinants}

\section{Properties of Determinants}

\section{Cramer's Rule, Volume, and Linear Transformations}




\chapter{Vector Spaces}

\section{Vector Spaces and Subspaces}

\section{Null Spaces, Column Spaces, and Linear Transformations}

\section{Linearly Independent Sets and Bases}

\section{Coordinate Systems}

\section{The Dimension of a Vector Space}

\section{Rank}

\section{Change of Basis}




\chapter{Eigenvectors}

\section{Eigenvectors and Eigenvalues}

\section{The Characteristic Equation}

\section{Diagonalization}




\end{document}
.