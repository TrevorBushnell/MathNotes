
\documentclass{package/notes}
\usepackage[english]{babel}
\usepackage{amssymb,amsmath,amsfonts}  %%% for maths
%%%%%%%%%%%%%%%%%%%%%%%%%%%%%%%%%%%%%
\usepackage{package/color-env}
\usepackage{lipsum}
\renewcommand\qedsymbol{$\blacksquare$}
%%%%%%%%%%%%%%%%%%%%%%%%%%%%%%%%%%%%%

\begin{document}

	\begin{titlepage} % Suppresses headers and footers on the title page
		
		\centering % Centre everything on the title page
		
		\scshape % Use small caps for all text on the title page
		
		\vspace*{\baselineskip} % White space at the top of the page
		
		%------------------------------------------------
		%	Title
		%------------------------------------------------
		
		\rule{\textwidth}{1.6pt}\vspace*{-\baselineskip}\vspace*{2pt} % Thick horizontal rule
		\rule{\textwidth}{0.4pt} % Thin horizontal rule
		
		\vspace{0.75\baselineskip} % Whitespace above the title
		
		{\huge Multivariable Calculus\\} % Title
		
		\vspace{0.75\baselineskip} % Whitespace below the title
		
		\rule{\textwidth}{0.4pt}\vspace*{-\baselineskip}\vspace{3.2pt} % Thin horizontal rule
		\rule{\textwidth}{1.6pt} % Thick horizontal rule
		
		\vspace{2\baselineskip} % Whitespace after the title block
		
		%------------------------------------------------
		%	Subtitle
		%------------------------------------------------
		
		\LARGE{Public Notes for Any Multivariable Calculus Course} 
		
		\vspace*{3\baselineskip} % Whitespace under the subtitle
		
		
		
		\vspace{0.5\baselineskip} 
		
		\vspace{0.5\baselineskip} 
		
		
		\vfill 
		
		%------------------------------------------------
		% Author
		%------------------------------------------------
		
		
		\vspace{0.3\baselineskip} 
		
		
		{\large Edited by\\  Trevor Bushnell} 
		
	\end{titlepage}
	\tableofcontents
%\newpage
\chapter{Vectors and the Geometry of Space}

\section{3D Coordinate Systems}

\subsection{About 3D Coordinate Systems}

\subsection{The Distance Formula (3D)}

\subsection{Equation of a Sphere}



\section{Vectors}

\subsection{About Vectors}

\subsection{Vector Components and Magnitudes}

\subsection{Adding Vectors}

\subsection{Other Properties of Vectors}

\subsection{Basis Vectors}

\subsection{Applications}



\section{The Dot Product}

\subsection{Definition of the Dot Product}

\subsection{What Does the Dot Product Represent?}

\subsection{Direction Angles and Direction Cosines}



\section{The Cross Product}

\subsection{Definition of the Cross Product}

\subsection{What Does the Cross Product Represent?}

\subsection{Scalar Triple Products}

\subsection{Application: Torque}




\section{Equations of Lines and Planes}

\subsection{Lines in 3D}

\subsection{Planes}




\chapter{Partial Derivatives}



\chapter{Multiple Integrals}



\chapter{Vector Calculus}




\end{document}
.