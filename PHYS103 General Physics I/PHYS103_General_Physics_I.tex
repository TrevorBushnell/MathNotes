\documentclass{package/notes}
\usepackage[english]{babel}
\usepackage{amssymb,amsmath,amsfonts}  %%% for maths
%%%%%%%%%%%%%%%%%%%%%%%%%%%%%%%%%%%%%
\usepackage{package/color-env}
\usepackage{lipsum}
\usepackage{graphicx}
\renewcommand\qedsymbol{$\blacksquare$}

\newcommand{\F}{\sum F}
\newcommand{\m}{\text{ m}}
\newcommand{\km}{\text{ km}}
\newcommand{\cm}{\text{ cm}}
\newcommand{\s}{\text{ s}}
\newcommand{\kg}{\text{ kg}}
\newcommand{\N}{\text{ N}}
\newcommand{\J}{\text{ J}}
%%%%%%%%%%%%%%%%%%%%%%%%%%%%%%%%%%%%%

\begin{document}

	\begin{titlepage} % Suppresses headers and footers on the title page
		
		\centering % Centre everything on the title page
		
		\scshape % Use small caps for all text on the title page
		
		\vspace*{\baselineskip} % White space at the top of the page
		
		%------------------------------------------------
		%	Title
		%------------------------------------------------
		
		\rule{\textwidth}{1.6pt}\vspace*{-\baselineskip}\vspace*{2pt} % Thick horizontal rule
		\rule{\textwidth}{0.4pt} % Thin horizontal rule
		
		\vspace{0.75\baselineskip} % Whitespace above the title
		
		{\huge PHYS103: General Physics I Notes} % Title
		
		\vspace{0.75\baselineskip} % Whitespace below the title
		
		\rule{\textwidth}{0.4pt}\vspace*{-\baselineskip}\vspace{3.2pt} % Thin horizontal rule
		\rule{\textwidth}{1.6pt} % Thick horizontal rule
		
		\vspace{2\baselineskip} % Whitespace after the title block
		
		%------------------------------------------------
		%	Subtitle
		%------------------------------------------------
		
		\LARGE{} 
		
		\vspace*{3\baselineskip} % Whitespace under the subtitle
		
		
		
		\vspace{0.5\baselineskip} 
		
		{\scshape   \LARGE Professor David James\\ } % Editor list
		
		\vspace{0.5\baselineskip}
		
		\vfill 
		
		%------------------------------------------------
		% Author
		%------------------------------------------------
		
		
		\vspace{0.3\baselineskip} 
		
		
		{\large Edited by\\  Trevor Bushnell} 
		
	\end{titlepage}
	\tableofcontents
%\newpage
\chapter*{Introduction}

This document aims to highlight the important content of the PHYS103 course in traditional notes format. These notes are completely open-source, which means anyone is allowed to use these notes for their own personal benefit without having to seek permission from myself. \newline

While these notes are designed for the PHYS103 course, all of the content seen in these notes are equivalent to a one-semester calculus-based introductory physics class taken at many universities. The content in these notes might also have some overlap with the AP Physics C: Mechanics Course. As such, students in the AP Physics C: Mechanics course and students taken any calculus-based introductory physics course might still find the content provided in these notes useful.\newline

Due to the open-source nature of these notes, anyone is allowed to contribute to improving these notes as they see fit. Since I am using GitHub to distribute these notes easily, you must request all changes through the repository website on GitHub, which you can find \textbf{here}. If you are interested in contributing to these notes, then there are a few ways that you can do so:\newline

\begin{enumerate}
	\item \textbf{Open and submit an issue on my GitHub repository:} I write all my notes in \LaTeX, which is a typesetting language that is really helpful when it comes to typing and rendering math equations quickly and easily. If you do not know how to write \LaTeX code but are still interested in making a change to the notes, you can open an issue by going to the MathNotes repo on GitHub, and clicking on the button labeled "New Issue." From there, you can type out the change that you wish to see in the notes. It would be helpful if you would indicate what course you would like to see changed so that I can understand what you are referring to. I will then update the code to include your issue so that you don't have to worry about writing the code yourself.
	\item \textbf{Create and submit a pull request:} If you know how to write LaTeX code and you understand how GitHub works, you can submit a pull request where you can write the code that you want to change yourself. I will then review the code and either submit the code to be incorporated into the notes OR provide some comments on your code if I wish for something to be different. 
\end{enumerate}\newpage

Thank you so much for using these notes. I hope that the information is provided in such a way that it can help you when reviewing content for you homework, quizzes, and exams and just in general when it comes to learning the content for the course. Happy studying!

\newpage


\chapter{Measurement}




\chapter{1D Motion}


\chapter{Vectors and Vector Notation}


\chapter{2D Motion}


\chapter{Forces}

\section{Newton's Second Law}


\end{document}