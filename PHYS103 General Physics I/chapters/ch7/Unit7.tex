\section{General Fluid Statics Terms}

\begin{itemize}
    \item \textbf{FLUID:} A substance that accompanies chips and salsa
    \item \textbf{DENSITY:} $m = \rho V$
    \begin{itemize}
        \item The denisity of water is $\rho_{water} = 1000 \frac{\kg}{\m^3} = 1 \frac{\g}{\cm^3}$
    \end{itemize}
    \item \textbf{COMMON VOLUME UNITS:} Liter, Cubic Meter
    \begin{itemize}
        \item $1 \text{ Liter } = 1000 \text{ cm}^3$
        \item $1000 \text{ Liter } = 1 \text{ m}^3$
    \end{itemize}
    \item \textbf{PRESSURE:} $P = \frac{F}{A_\perp}$
    \begin{itemize}
        \item PRESSURE AT SEA LEVEL: $14.7 \psii = 1.01 \times 10^5 \pa = 101 \text{ KPa}$
    \end{itemize}
\end{itemize}

\begin{problem}[Basic Pressure Problem]
    A room is $3.5\m \times 4.2\m \times 2.4\m$. If the density of air is $1.21 \frac{\kg}{\m^3}$, what is the force weight in the room and the pressure at the floor of the room?

    \[
    \begin{aligned}
        W &= mg\\
        W &= \rho_{air}V_{room}g\\
        W &= (1.21)(3.5)(4.2)(2.4)(9.81)\\
        W &= 418 \N 
    \end{aligned}    
    \]

    \[
    \begin{aligned}
        P &= \frac{F}{A}\\
        P &= \frac{\rho_{air}lwhg}{wl}\\
        P &= \rho_{air}hg\\
        P &= (1.21)(2.4)(9.81)\\
        P &= 28.44 \pa
    \end{aligned}    
    \]
\end{problem}\newpage

\begin{definition}[General Pressure Equations]{def7.1.1:label}
    The pressure at a given reference point $R$ is given by the following equation:

    \[
    \begin{aligned}
        P = P_R \pm \rho Hg
    \end{aligned}    
    \]

    If you want to find the \textbf{absolute pressure} at a point $R$, then you can use the following equation:

    \[
    P = P_{atm} \pm \rho Hg    
    \]

    The \textbf{guage pressure} is the pressure experience without considering atmospheric pressure, which is given by the following equation:

    \[
    \begin{aligned}
        P - P_{atm} = \rho Hg
    \end{aligned}    
    \]
\end{definition}


\begin{problem}
    There is a pipe that is in the shape of a square D without the top. Water is put in the pipe so that the water is at the same height on both sides of the pipe. Oil is placed on the left side of the pipe and the height of the water rises on the right. What is the density of the oil?

    \[
    \begin{aligned}
        P_{atm} + \rho_{oil}gD = P_{atm} + \rho_{water}gy_1\\
        \rho_{oil} = \rho_{water}\frac{y_1}{D}
    \end{aligned}    
    \]
\end{problem}

\begin{itemize}
    \item A force on an imcompressible fluid is additive - that is if you enact a force down on a fluid, you add the created pressure to the reference pressure of the point in the fluid you wish to analyze.
\end{itemize}


\section{Archimedes' Principle}

\begin{definition}[Archimedes' Principle]{def7.2.1:label}
    An object that is submerged in water is equal to the weight of the displaced fluid. The weight of the displaced fluid is called the \textbf{buoyant force}.

    \[
    F_B = V\rho g    
    \]
\end{definition}

\begin{problem}[The Concrete Canoe]
    \[
    \begin{aligned}
        \sum F &= 0 = F_B - mg\\
        F_B &= mg\\
        \rho_{water}V_{disp}g &= \rho_{obj}V_{obj}g
    \end{aligned}    
    \]
\end{problem}


\begin{problem}[The (Floating?) Rock]
    \[
    \begin{aligned}
        \sum F_y &= 0 = F_B + N - W\\
        N &= W - F_B\\
        N &= \rho_RVg - \rho_WVg\\
        N &= Vg(\rho_R-\rho_W)
    \end{aligned}    
    \]
\end{problem}

\begin{problem}[Floating Wood]
    \[
    \begin{aligned}
        \sum F_y &= F_B - W\\
        F_B &= W\\
        \rho_{H_20}(H-y)wlhg &= \rho_Whwlg\\
        \rho_{H_20}H - \rho_{H_20}y &= \rho_Wh\\
        y &= h(\rho_{H_20} - \rho_W)\\
        y &= \frac{H(\rho_{H_20}-\rho_W)}{\rho_{H_20}}
    \end{aligned}    
    \]
\end{problem}

\begin{problem}[Dams (Damn)]
    \[
    \begin{aligned}
        dF &= P(y)W(y)dy\\
        P(y) &= P_r + \rho g y\\
        \int_{F = 0}^{F_{net}}dF &= \int_0^{100}(\rho gy)\left(W-\frac{y}{2}\right)\\
        \rho g \int_0^{100}\left(Wy - \frac{y^2}{2}\right)dy &=
    \end{aligned}     
    \]
\end{problem}